\documentclass{article}

% if you need to pass options to natbib, use, e.g.:
% \PassOptionsToPackage{numbers, compress}{natbib}
% before loading nips_2017
%
% to avoid loading the natbib package, add option nonatbib:
% \usepackage[nonatbib]{nips_2017}


\usepackage[final]{nips_2017}
% to compile a camera-ready version, add the [final] option, e.g.:
% \usepackage[final]{nips_2017}

\usepackage[utf8]{inputenc} % allow utf-8 input
\usepackage[T1]{fontenc}    % use 8-bit T1 fonts
\usepackage{hyperref}       % hyperlinks
\usepackage{url}            % simple URL typesetting
\usepackage{booktabs}       % professional-quality tables
\usepackage{amsfonts}       % blackboard math symbols
\usepackage{nicefrac}       % compact symbols for 1/2, etc.
\usepackage{microtype}      % microtypography
\usepackage{graphicx}
\usepackage{amsmath}
\usepackage{listings}
\usepackage{xcolor}
\usepackage{float}


\usepackage[
backend=bibtex,
sorting=none,
]{biblatex}
\addbibresource{bib.bib}

%\usepackage{algorithm, algorithmic}%
\usepackage{tabularx,dcolumn,booktabs,caption,ragged2e,geometry}
\newcolumntype{d}[1]{D..{#1}} % for aligning numerical data on decimal marker
\newcommand\mX[1]{\multicolumn{1}{X}{#1}} % handy shortcut macro
\newcolumntype{C}{>{\Centering\arraybackslash}X}
\newcommand\mC[1]{\multicolumn{1}{C}{\textbf{#1}}} % another handy shortcut macro
\raggedbottom
\setlength\floatsep{1.25\baselineskip  minus 3pt}
\setlength\textfloatsep{1.25\baselineskip  minus 3pt}
\setlength\intextsep{1.25\baselineskip  minus 3pt}
\usepackage{algorithmicx}
\usepackage[Algorithm,ruled]{algorithm}
\lstset { %
    language=python,
    backgroundcolor=\color{black!5}, % set backgroundcolor
    basicstyle=\footnotesize,% basic font setting
}
%% # PROBABILITY

\newcommand{\g}{\,|\,}
\renewcommand{\gg}{\,\|\,}
\DeclareRobustCommand{\KL}[2]{\ensuremath{\textrm{KL}\left(#1\;\|\;#2\right)}}

% # DISTRIBUTIONS

\newcommand{\Gam}{\textrm{Gam}}
\newcommand{\InvGam}{\textrm{InvGam}}

% # MISCELLANEOUS

\renewcommand{\d}[1]{\ensuremath{\operatorname{d}\!{#1}}}
\newcommand{\diag}{\textrm{diag}}
\newcommand{\supp}{\textrm{supp}}
\DeclareMathOperator*{\argmax}{arg\,max}
\DeclareMathOperator*{\argmin}{arg\,min}
\newcommand\indep{\protect\mathpalette{\protect\independenT}{\perp}}
\def\independenT#1#2{\mathrel{\rlap{$#1#2$}\mkern2mu{#1#2}}}

% # BOLD MATHEMATICS

\newcommand{\mba}{\mathbold{a}}
\newcommand{\mbb}{\mathbold{b}}
\newcommand{\mbc}{\mathbold{c}}
\newcommand{\mbd}{\mathbold{d}}
\newcommand{\mbe}{\mathbold{e}}
%\newcommand{\mbf}{\mathbold{f}}
\newcommand{\mbg}{\mathbold{g}}
\newcommand{\mbh}{\mathbold{h}}
\newcommand{\mbi}{\mathbold{i}}
\newcommand{\mbj}{\mathbold{j}}
\newcommand{\mbk}{\mathbold{k}}
\newcommand{\mbl}{\mathbold{l}}
\newcommand{\mbm}{\mathbold{m}}
\newcommand{\mbn}{\mathbold{n}}
\newcommand{\mbo}{\mathbold{o}}
\newcommand{\mbp}{\mathbold{p}}
\newcommand{\mbq}{\mathbold{q}}
\newcommand{\mbr}{\mathbold{r}}
\newcommand{\mbs}{\mathbold{s}}
\newcommand{\mbt}{\mathbold{t}}
\newcommand{\mbu}{\mathbold{u}}
\newcommand{\mbv}{\mathbold{v}}
\newcommand{\mbw}{\mathbold{w}}
\newcommand{\mbx}{\mathbold{x}}
\newcommand{\mby}{\mathbold{y}}
\newcommand{\mbz}{\mathbold{z}}

\newcommand{\mbA}{\mathbold{A}}
\newcommand{\mbB}{\mathbold{B}}
\newcommand{\mbC}{\mathbold{C}}
\newcommand{\mbD}{\mathbold{D}}
\newcommand{\mbE}{\mathbold{E}}
\newcommand{\mbF}{\mathbold{F}}
\newcommand{\mbG}{\mathbold{G}}
\newcommand{\mbH}{\mathbold{H}}
\newcommand{\mbI}{\mathbold{I}}
\newcommand{\mbJ}{\mathbold{J}}
\newcommand{\mbK}{\mathbold{K}}
\newcommand{\mbL}{\mathbold{L}}
\newcommand{\mbM}{\mathbold{M}}
\newcommand{\mbN}{\mathbold{N}}
\newcommand{\mbO}{\mathbold{O}}
\newcommand{\mbP}{\mathbold{P}}
\newcommand{\mbQ}{\mathbold{Q}}
\newcommand{\mbR}{\mathbold{R}}
\newcommand{\mbS}{\mathbold{S}}
\newcommand{\mbT}{\mathbold{T}}
\newcommand{\mbU}{\mathbold{U}}
\newcommand{\mbV}{\mathbold{V}}
\newcommand{\mbW}{\mathbold{W}}
\newcommand{\mbX}{\mathbold{X}}
\newcommand{\mbY}{\mathbold{Y}}
\newcommand{\mbZ}{\mathbold{Z}}

\newcommand{\mbalpha}{\mathbold{\alpha}}
\newcommand{\mbbeta}{\mathbold{\beta}}
\newcommand{\mbdelta}{\mathbold{\delta}}
\newcommand{\mbepsilon}{\mathbold{\epsilon}}
\newcommand{\mbchi}{\mathbold{\chi}}
\newcommand{\mbeta}{\mathbold{\eta}}
\newcommand{\mbgamma}{\mathbold{\gamma}}
\newcommand{\mbiota}{\mathbold{\iota}}
\newcommand{\mbkappa}{\mathbold{\kappa}}
\newcommand{\mblambda}{\mathbold{\lambda}}
\newcommand{\mbmu}{\mathbold{\mu}}
\newcommand{\mbnu}{\mathbold{\nu}}
\newcommand{\mbomega}{\mathbold{\omega}}
\newcommand{\mbphi}{\mathbold{\phi}}
\newcommand{\mbpi}{\mathbold{\pi}}
\newcommand{\mbpsi}{\mathbold{\psi}}
\newcommand{\mbrho}{\mathbold{\rho}}
\newcommand{\mbsigma}{\mathbold{\sigma}}
\newcommand{\mbtau}{\mathbold{\tau}}
\newcommand{\mbtheta}{\mathbold{\theta}}
\newcommand{\mbupsilon}{\mathbold{\upsilon}}
\newcommand{\mbvarepsilon}{\mathbold{\varepsilon}}
\newcommand{\mbvarphi}{\mathbold{\varphi}}
\newcommand{\mbvartheta}{\mathbold{\vartheta}}
\newcommand{\mbvarrho}{\mathbold{\varrho}}
\newcommand{\mbxi}{\mathbold{\xi}}
\newcommand{\mbzeta}{\mathbold{\zeta}}

\newcommand{\mbDelta}{\mathbold{\Delta}}
\newcommand{\mbGamma}{\mathbold{\Gamma}}
\newcommand{\mbLambda}{\mathbold{\Lambda}}
\newcommand{\mbOmega}{\mathbold{\Omega}}
\newcommand{\mbPhi}{\mathbold{\Phi}}
\newcommand{\mbPi}{\mathbold{\Pi}}
\newcommand{\mbPsi}{\mathbold{\Psi}}
\newcommand{\mbSigma}{\mathbold{\Sigma}}
\newcommand{\mbTheta}{\mathbold{\Theta}}
\newcommand{\mbUpsilon}{\mathbold{\Upsilon}}
\newcommand{\mbXi}{\mathbold{\Xi}}

\newcommand{\mbzero}{\mathbold{0}}
\newcommand{\mbone}{\mathbold{1}}
\newcommand{\mbtwo}{\mathbold{2}}
\newcommand{\mbthree}{\mathbold{3}}
\newcommand{\mbfour}{\mathbold{4}}
\newcommand{\mbfive}{\mathbold{5}}
\newcommand{\mbsix}{\mathbold{6}}
\newcommand{\mbseven}{\mathbold{7}}
\newcommand{\mbeight}{\mathbold{8}}
\newcommand{\mbnine}{\mathbold{9}}



\title{Title}

\author{
Author Name \\
Department of Computer Science\\
Columbia University \\
\texttt{name@columbia.edu} \\
\And
Jason Krone \\
Department of Computer Science\\
Columbia University \\
\texttt{jpk2151@columbia.edu} \\
}

\begin{document}

\maketitle
\begin{abstract}
This project builds on the work done in "One-Shot Visual Imitation Learning via Meta-Learning"
by Chelsea Finn et al. 2017. We aim to both duplicate the results presented in the paper
for simulated environements as well as improve the neural network architecture proposed
by Finn et al. through the addition of a memory component. We evaluate of our model architecture
on the same simulated robotic environments used by Finn et al. to allow for direct comparison
between the two models.
\end{abstract}




\section{Introduction}

Introduce the problem statement


\section{Related Work}
There are a number of works that focus on the broad notion of meta learning and memory networks.
However the most important Meta-Learning papers are those that use the Model-Agnostic Meta-Learning (MAML)
framework that underlies MIL. MAML was introduced by Finn et al. in "Model-Agnostic Meta-Learning for
Fast Adaptation of Deep Networks". Subsequently, it was modified
for application to imitation learning in "One-Shot Visual Imitation Learning via Meta-Learning". 
Most recently Finn et al. published "One-Shot Imitation from Observing Humans via Domain-Adaptive Meta-Learning", which extends
a method for imitation learning using only video footage 
i.e. without knowledge of the experts actions.

Similarly, there are a large number of papers that apply memory networks. Therefore, we focus on
works that introduce new types of memory modules. The idea of a memory module was first introduced in
the paper "Neural Turing Machines". Later this concept was improved in "Hybrid computing using a neural network with dynamic external memory" 
through the addition of dynamic memory allocation and de-allocation for memory reuse. Since then
new types of memory networks have been proposed for few shot learning and meta learning. Specifically, 
"Learning to Remember Rare Events" and "Matching Networks for One Shot Learning" both apply
differentiable memory modules for few shot learning on the Omniglot dataset.


\section{Approach}

Meta-Imitation Learning optimizes the following objective objective:
$$\min_{\theta} \sum_{T_i \sim p(T)} L_{T_i}(f_{\theta_{i}^{'}}) =
\sum_{T_i \sim p(T)} L_{T_i}(f_{\theta - \alpha \Delta_{\theta}L_{T_i}(f_\theta)})$$
, where $L_{T_i}$ is defined as
$$L_{T_i}(f_\phi) = \sum_{\tau^{(j) \sim T_i}} \sum_t \norm{f_{\phi}(o_t^{(j)}) - a_t^{(j)}}_2^2$$





\section{Experiments}

What experiments did you run? What were your hyperparameters / recipes ? Everything needed to recreate the results


\section{Milestones}
\begin{enumerate}
    \item Set up the three simulated environments 
    \item Get MIL running on one of the three environemtns
    \item Create script to evalute success and plot performance
    \item Train and evalute model on all environements
    \item Download reference implementations of "Learning to Remember Rare Events" 
          (LRRE) and "Matching Networks for One Shot Learning" (Matching Networks)
    \item Reproduce LRRE results on Omniglot
    \item Reproduce Matching Networks Results on Omniglot
    \item Attach LRRE memory module to MIL network architecture 
    \item Attach Matching Networks memory module to MIL network architecture
    \item Evaluate results both memory networks on all three environments
\end{enumerate}


\clearpage

\nocite{*}
\printbibliography

\end{document}
